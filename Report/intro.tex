COVID-19 has taken the world by storm in the year 2020. It has wreaked havoc in lives of everyone on the planet and re-defined human interaction. It also has caused the death of more than a million people across the planet. It is in these trying times that the impact of technology \& the responsibility on it increase multi-fold. In particular, data visualization has a large role to play due to its ability to present the results of scientific analysis to the general public. A good visualization can present trends in data which might not be visible from numbers and help the layman appreciate the urgency of the situation better. It can help ensure that each additional life lost due to COVID is not just another number added to the total, but actually serves the purpose of increasing the awareness in others.

There is already a plethora of websites online detailing numbers related to COVID-19 and quite a few of those have very interactive data visualizations to help better analyze the data. A majority of them focus on presenting the data in pie/bar chart form which provides for good viewing of data. However, this does not help to solve the inherent issue with data bias. For ex, just looking at the raw data in terms of COVID cases does not do justice to the impact of population density, per capita income and other demographics on the spread. It is important to take all the factors into account while presenting the visualization of the case spread. This would aid in creating better awareness about the ease of COVID spread.

Due to the interdependence of data, it is important to look at the correlation to all metrics. A correlation plot like a small multiple plot can help to present this interdependence of data better. Adding interaction helps the user to choose and view the data of his/her choice. Another very useful approach is to have a cartographic map of this correlated data to help people analyze the data for the regions of their interest. Some might think that Cartography is showing the map using different colors and make them look visually stunning but Cartography is both science and art form of study of maps. Choropleth maps help to better visualize the data by showing the relative scores for different regions of interest.

With the rise of COVID cases in the USA due to the third wave, there is an increased emphasis on the need for better visualization tools to create further awareness among the general public. The project concentrates on how to apply the dataset of COVID cases on the map of USA based on both the raw data and per capita cases using its spatial information and its layout. As we are integrating the non-spatial data such as number of people recovered, regions or states mostly affected, people who have died from the virus and many other contexts with base spatial data i.e., map of the USA, this project will be based on choropleth mapping techniques.

The objectives of this project are:
\begin{itemize}
  \item to perform a comprehensive analysis of the existing data visualization techniques for COVID-19
  \item to develop a choropleth map based visualization tool to better visualize COVID-19 spread and its correlation to different parameters like per capita
  \item to provide a better tool to the general public and help to create more awareness via these visualizations
\end{itemize}
