% Impact - 20%
\section{Impacts}
\label{sec:impact}

Though there are a ton of COVID related visualization tools \& websites, but the aim of this project is to create a visualization tool that will help users to derive correlation data between different metrics using cumulative plots for easy comparison and present the complex data in a simplified manner by helping with an option to check the covid spread based on per-capita to the general public.  
\subsection{Conclusion}
\begin{itemize}
\item Anyone using this tool should be able to get a feel for the general state of covid spread in the region that they are in through choropleth mapping techniques. 
\item The implementation of interactive visualization implemented in this project aids in better understanding the correlation between different factors impacting the spread of Covid. 
\item By providing an option to showcase per-capita data for each state and as well as compare between with and without percapita scenarios, this project helps to show the actual severity/recovered scenario of the states which is rare to find in other similar implementation.
\item Multi-data handling can be added to also show county data along with state wise data. However, this will have to take care of inherent data bias since the data sources for the two are different and ensuring proper weights to each data will be crucial.
\end{itemize}

\subsection{Future improvement}
One avenue to improve the visualization tool presented here is in handling the data. While this tool can handle data on the fly, more exception handling can be added to handle cases where numbers might not have been reported by individual states. Addition of slider based animation tools to show cumulative information on choropleth map is another possible improvement.

